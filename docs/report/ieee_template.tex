% Minimal IEEE-style skeleton for CA1 report
\documentclass[conference]{IEEEtran}
\usepackage{graphicx}
\usepackage{url}
\usepackage{booktabs}
\usepackage{amsmath}
\usepackage{hyperref}

\title{Emotion and Topic Dynamics Around a Climate Policy Announcement: Interpretable Baselines (CA1)}

\author{\IEEEauthorblockN{Group Name Placeholder}\\
\IEEEauthorblockA{University / Institution\\ Email@example.com}}

\begin{document}
\maketitle

\begin{abstract}
We present baseline analyses of emotional and topical shifts in climate-policy news surrounding the IPCC AR6 Synthesis Report release. Using lexicon-based emotion detection, entity-centric sentiment, n-gram shift metrics, PMI collocations, and TF-IDF classification, we quantify changes pre vs post event and set the stage for later contextual embedding extensions.
\end{abstract}

\section{Introduction}
Motivation, research question, why pre/post analysis matters.

\section{Related Work}
Brief on emotion arcs, media framing, policy event analyses.

\section{Data Collection}
GDELT query, keywords, time windows, filtering criteria, ethical considerations.

\section{Methods}
Lexicons (NRC, VADER), preprocessing (spaCy), topic modeling (LDA), collocations (PMI), BoW classification, evaluation metrics.

\section{Results}
Figures: emotion arc, bar chart, topic heatmap, LSA scatter. Tables: period emotion rates, n-gram shifts, collocations, classifier metrics.

\section{Discussion}
Interpretation of shifts, limitations (lexicon coverage, small sample), robustness checks.

\section{Future Work}
Contextual embeddings, dynamic topic modeling, semantic shift, stance/framing expansion.

\section{Conclusion}
Summary of baseline insights and extension path.

\section*{Acknowledgments}
(Optional)

\bibliographystyle{IEEEtran}
\begin{thebibliography}{99}
\bibitem{nrc} S. Mohammad and P. Turney, ``Crowdsourcing a Word-Emotion Association Lexicon,'' *Computational Intelligence*, 2013.
\bibitem{vader} C. Hutto and E. Gilbert, ``VADER: A Parsimonious Rule-based Model for Sentiment Analysis of Social Media Text,'' *ICWSM*, 2014.
\bibitem{gdelt} K. Leetaru, ``The GDELT Project,'' 2013.\url{https://www.gdeltproject.org/}
\end{thebibliography}

\end{document}
